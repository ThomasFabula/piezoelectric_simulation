% Datei: AKT-TOM.TEX
% Formeln zu der Aktoren-Abschätzung in LaTeX

\section[Aktoren]
{Analytische Beschreibung mikromechanischer piezoelektrischer Bimorphstrukturen}


Die symmetrische Übertragungsmatrix [$e$] beschreibt den Biegezustand
des Balkens bei unterschiedlichen mechanischen und elektrischen
Belastungen und die verknüpft internen mit den externen Parametern:
%
\begin{eqnarray*}
  \left(
  \begin{array}{c}
  \alpha \\ \delta \\ \upsilon \\ Q
  \end{array} \right) & = &
  \left[
  \begin{array}{llll}
  e_{11} & e_{12} & e_{13} & e_{14} \\
  e_{21} & e_{22} & e_{23} & e_{24} \\
  e_{31} & e_{32} & e_{33} & e_{34} \\
  e_{41} & e_{42} & e_{43} & e_{44}
  \end{array}
 \right]
 \left(
 \begin{array}{c}
 M \\ F \\ p \\ U
 \end{array}
\right)
\end{eqnarray*}
%
Für die Komponenten $\displaystyle \frac{K}{A} \cdot e_{ij}$ der Matrix gilt:
%
\begin{eqnarray*}
%e & = & A \cdot
\left [
\begin{array}{cccc}
\displaystyle \frac{12L}{Kb} & \displaystyle \frac{6L^{2}}{Kb} &
\displaystyle \frac{2L^{3}}{K} & \frac{6d_{31}BL}{K} \\
\\
\displaystyle \frac{6L^{2}}{Kb} & \displaystyle \frac{4L^{3}}{Kb} &
\displaystyle \frac{3L^{4}}{2K} &
               \displaystyle \frac{3d_{31}BL^{2}}{K}\\
\\
\displaystyle \frac{2L^{3}}{K} & \displaystyle \frac{3L^{4}}{2K} &
\displaystyle \frac{3L^{5}b}{5K} & \displaystyle
\frac{3d_{31}BL^{3}b}{K}\\
\\
\displaystyle \frac{6d_{31}BL}{K}     &
\displaystyle \frac{3d_{31}BL^{2}}{K} &
\displaystyle \frac{d_{31}BL^{3}b}{K} &
\quad \displaystyle \frac{Lb}{Ah_{p}} % \cdot
\left( \epsilon^{K \sigma}_{33} -
%\frac{
d_{31}^{2} h_{Si} (s^{Si}_{11} h_{p}^{3} +
s^{p}_{11} h_{Si}^{3}  \right)
\end{array}
\right ]
\end{eqnarray*}
%
mit den Abküzungen $ A, B, K $:
%
\begin{eqnarray*}
   A & = & S^{Si}_{11} S^{p}_{11} \, \left( S^{p}_{11} h_{Si} \, + \,
           S^{Si}_{11} h_{p} \right ) \\
   B & = & \frac{h_{Si} (h_{Si} \, + \, h_{p}) }{ S^{p}_{11} h_{Si} \,
               +  \, S^{Si}_{11} h_{p}} \\
     &   &   \nonumber  \\
   K & = & (S^{Si}_{11})^{2} \, (h_{p})^{4} \, + \, 4S^{Si}_{11}
                   S^{p}_{11} h_{Si}(h_{p})^{3} \nonumber \\
     &   & + \, 6S^{Si}_{11} S^{p}_{11}(h_{Si})^{2}(h_{p})^{2} \, + \,
                4S^{Si}_{11} S^{p}_{11} h_{p} (h_{Si})^{3} \nonumber \\
     &   & + \, (S^{p}_{11})^{2}(h_{Si})^{4}
 \end{eqnarray*}
%
wobei:
\begin{eqnarray*}
 L, b                      & : &
  \mbox{L„nge, Breite des Biegebalkens [m]} \\
 h_{Si},h_{p}              & : &
  \mbox{Substratdicke (hier Silizium), Piezoschichtdicke [m]}   \\
 S^{Si}_{11}, S^{p}_{11}   & : &
  \mbox{Steifigkeitskoeffizient des Substrats, der Piezoschicht}
  \quad \left [ \frac{m^{2}}{N} \right ] \\
 d_{31} & : & \mbox{transversaler piezoelektrischer Koeffizient}
  \left [ \frac{C}{N} \right ] \\
 \epsilon^{\sigma}_{33}    & : &
  \mbox{Dielektrizit#tskonstante in E-Feldrichtung}
  \left [ \frac{C}{Vm}\right ]
\end{eqnarray*}

